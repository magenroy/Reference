\documentclass{article}
\usepackage{amsmath, amssymb}
\title{Miscellaneous Math Notes}
\author{Roy Magen}
\date{ }
\begin{document}
	
	\begin{titlepage} \maketitle \end{titlepage}

	\paragraph{Complex Conjugate}
		The complex number $ z = a + ib $ has conjugate $ \overline{z} = a - ib
		$ where $ a,b \in \mathbb{R} $.
		\begin{gather}
			z \overline{z} = (a + ib)(a - ib) = a^2 + b^2 \\
			\overline{re^{i \theta}} = re^{-i \theta}
		\end{gather}

	\paragraph{Transpose of a Matrix}
		The transpose of the $ n \times m $ matrix $ A $ is the $ m \times n $ matrix $ A^T $ created by one of the
		following operations
		\begin{itemize}
			\item reflecting $ A $ over its main diagonal
			\item swapping the rows and columns of $ A $
		\end{itemize}
		$ A_{ij} = \left( A^T \right)_{ji} $ where the
		element $ \text{matrix}_{ab} $ is the entry at row $ a $
		and column $ b $ of the matrix.
		\begin{gather}
			A \text{ is an } n \times m \text{ matrix } \iff A^T \text{ is an }
			m \times n \text{ matrix.}
		\end{gather}

	\paragraph{Conjugate Transpose}
		The conjugate transpose of the $ n \times m $ matrix $ A $ is the $ m
		\times n $ matrix $ A^* $ formed by taking the complex conjugate of each
		entry of the transpose matrix $ A^T $
		\begin{gather}
			A^* = \left( \overline{A} \right)^T = \overline{A^T}
		\end{gather}
	
	\paragraph{Unitary and Orthogonal Matrices}
		A square matrix $ U $ is unitary if $ U^*U = UU^* = I $ where $ I $ is
		the identity matrix. \\
		If $ U $ has only real entries, it is an orthogonal matrix, $ U^TU =
		UU^T = I $.

\end{document}
